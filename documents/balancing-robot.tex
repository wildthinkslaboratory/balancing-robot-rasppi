\documentclass[]{article}
\usepackage[utf8]{inputenc}
\usepackage{multicol}
\usepackage{amsmath}
\usepackage{amsthm}
\usepackage{bm}

% this just sets up the page size and margins
\setlength{\topmargin}{0pt}
\setlength{\oddsidemargin}{0pt}
\setlength{\evensidemargin}{0pt}
\setlength{\textwidth}{6.5in}
\setlength{\textheight}{8.5in}
\setlength{\headheight}{0pt}

% create some short cut commands
\newcommand{\bx}{\boldsymbol{x}}
\newcommand{\by}{\boldsymbol{y}}
\newcommand{\bu}{\boldsymbol{u}}


% this is a bunch of stuff that allows python style code snippits
% Default fixed font does not support bold face
\DeclareFixedFont{\ttb}{T1}{txtt}{bx}{n}{10} % for bold
\DeclareFixedFont{\ttm}{T1}{txtt}{m}{n}{10}  % for normal
% Custom colors
\usepackage{color}
\definecolor{deepblue}{rgb}{0,0,0.5}
\definecolor{deepred}{rgb}{0.6,0,0}
\definecolor{deepgreen}{rgb}{0,0.5,0}

\usepackage{listings}

% Python style for highlighting
\newcommand\pythonstyle{\lstset{
		language=Python,
		basicstyle=\ttm,
		morekeywords={self},              % Add keywords here
		keywordstyle=\ttb\color{deepblue},
		emph={MyClass,__init__},          % Custom highlighting
		emphstyle=\ttb\color{deepred},    % Custom highlighting style
		stringstyle=\color{deepgreen},
		frame=tb,                         % Any extra options here
		showstringspaces=false,
		tabsize=4
}}

% Python environment
\lstnewenvironment{python}[1][]
{
	\pythonstyle
	\lstset{#1}
}
{}

% Python for inline
\newcommand\pythoninline[1]{{\pythonstyle\lstinline!#1!}}

\title{Balancing Robot}
\author{Izzy Mones and Heidi Dixon}
%\date{}

\begin{document}
	\maketitle

\section*{Robot Design}
Should have list of all the components. Maybe a picture. Do we need to show circuit stuff?

\section*{Model}
\subsection*{System Description (Lagrange's Method)}
Derive the equations here

\begin{eqnarray}
	\dot{x} & = & v \\
	\dot{v}          & = & \frac{-m^2L^2g \cos(\theta)\sin(\theta) + mL^2(mL\omega^2 \sin(\theta)-\delta v)+mL^2u }{mL^2(M+m(1-\cos(\theta)^2))} \\
	\dot{\theta}  &= & \omega \\
	\dot{\omega}  &= & \frac{(m+M)mgL\sin(\theta)-mL\cos(\theta)(mL\omega^2\sin(\theta)-\delta v)-mL\cos(\theta)u}{mL^2(M+m(1-\cos(\theta)^2))}
\end{eqnarray}
where $x$ is the cart position, $v$ is the velocity, $\theta$ is the pendulum angle, $\omega$ is the angular velocity, $m$ is the pendulum mass, $M$ is the cart mass, $L$ is the pendulum arm length, $g$ is the gravitational acceleration, $\delta$ is a friction damping on the cart, and $u$ is the control force applied to the cart.


\subsection*{Linearization}
To build a control system for our model we will linearize our system of equations around a fixed point $x_r$ where $x_r$ is the position where the robot is vertical, unmoving and positioned at the origin.  

The nonlinear system of differential equations
\begin{equation}
		\frac{d}{dt} \bx = f(\bx).
\end{equation}
can be represented as a Taylor series expansion around the point $\bx_r$.
\begin{equation}
	 f(\bx) = f(\bx_r) +\left.  \frac{\boldsymbol{df}}{\boldsymbol{dx}} \right|_{x_r}(\bx -\bx_r) + \left. \frac{\boldsymbol{d}^2\boldsymbol{f}}{\boldsymbol{dx}^2} \right|_{x_r}(\bx -\bx_r)^2 + \cdots
\end{equation}
Because $\bx_r$ is a fixed point, we know that $f(\bx_r) = 0$. Additionally, this approximation is only accurate in a small neighborhood around $\bx_r$.  In this neighborhood, we can assume that the value of $(\bx -\bx_r)$ is small, so higher order terms of this series will go to zero. So a fair estimate of our system is
\begin{equation}
	\frac{d}{dt} \bx \simeq \left.  \frac{\boldsymbol{df}}{\boldsymbol{dx}} \right|_{x_r}(\bx -\bx_r) 
\end{equation}
where $\left.  \frac{\boldsymbol{df}}{\boldsymbol{dx}} \right|_{x_r}$ is the Jacobian matrix for our system of equations $f(\bx)$ evaluated at the fixed point $\bx_r$. The Jacobian matrix for our system of equations evaluated at $\bx_r = [0\; 0\; \pi \; 0]$ is 

	\begin{equation}
		A = 
	\begin{bmatrix}
		0 & 1              & 0                          & 0 \\
		0 & -\frac{d}{M}      & \frac{mg}{M}                 & 0 \\
		0 & 0              & 0                          & 1 \\
		0 & -\frac{d}{ML} & -\frac{(m+M)g}{ML} & 0
	\end{bmatrix}
\end{equation}
\begin{equation*}
	\frac{d}{dt} \bx \simeq A(\bx -\bx_r)
\end{equation*}
\begin{equation}
	B = 
	\begin{bmatrix}
		0 \\
		\frac{1}{M}  \\
		0  \\
		\frac{1}{ML}   
	\end{bmatrix}
\end{equation}
\subsection*{LQR}
\subsection*{LQG}
\begin{itemize}
	\item Estimate the full state from sensor readings from the $x$ position, and the angular velocity $\omega$.
	\item Derive the Kalman filter matrix $K_f$ using the   \pythoninline{lqr} function from python control library.
	\begin{python}
	# This is our state disturbance matrix
	Vd = np.eye(4) 
	# This is our sensor noise matrix
	Vn = np.array([[1, 0], \
					[0, 1]])
	Kf = lqr(A.transpose(), C.transpose(), Vd, Vn)[0].transpose()
	\end{python}
	 \item To build the linear state space for our Kalman filter we build new matrices.
	\begin{itemize}
		\item $A_{kf} = A - K_fC$
		\item $B_{kf} = \begin{bmatrix} B & K_f \end{bmatrix}$
		\item $C = I_4$
		\item $D$ is a $0$ matrix with the same dimensions as $K_f$
	\end{itemize}
	These form a new linear system
	\begin{align}
		\frac{d}{dt} \bx  = & A_{kf} \bx + B_{kf} \bu \\
		\by = &C_{kf} \bx + D \bu
	\end{align}

	\item Our input vector $\bu = [u, x_s, \omega_s]$ is our motor torque $u$ and our two sensor readings, position $x_s$ and angular velocity $\omega_s$
	
\end{itemize}

\section*{Experiments}

\section*{Conclusions}
	




	
\end{document}